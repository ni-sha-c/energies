\documentclass{article}
\usepackage{amsmath,amssymb}
\begin{document}
The Chebyshev points for interpolating polynomials of maximum degree 
$n$:
\begin{align}
	x_j = \cos(j \pi/n), \; j=0,\cdots,n
\end{align}
The unique polynomial interpolant with values $\left\{u_j\right\}$ at the 
 Chebyshev points $\left\{x_j\right\}$:
\begin{align}
	p(x) = \sum_{j=0}^n u_j \prod_{k=0,k\neq j}^n  
	\dfrac{x-x_k}{x_j-x_k}
\end{align}
Define $q_j(x) = \prod_{i=0,i\neq j}^n (x - x_i)/(x_j - x_i) $, so 
that $p(x) = \sum_{j=0}^n u_j q_j(x)$. The derivative 
of $q_j$ at the Chebyshev points:
\begin{align}
	\label{eqn:DijRaw}
	q_j'(x) = \begin{cases} 
	\sum_{k=0,k\neq j}^n 
	\dfrac{1}{x_j - x_k} & x = x_j \\ 
	\dfrac{\prod_{k=0,k\neq j,i}^n (x_i - x_k)}
	{\prod_{k=0,k\neq j}^n (x_j - x_k)} & x = x_i, i \neq j 
	\end{cases}
\end{align}
The derivative of $p$, which is used to approximate the derivative of 
the function $u$, at the Chebyshev points:
\begin{align}
	p'(x_k) = \sum_{j=0}^n u_j q_j'(x_k).
\end{align}
Define the chebyshev differentiation matrix $D \in \mathbb{R}^{(n+1)\times (n+1)}$ as $D[i,j] = q_j'(x_i)$ so that 
$p'(x_i) = D[i,\cdot]\cdot U,$ where $U = \left\{u_j\right\}$. Using the following relationships between $x_i$, we can 
justify the simple program to calculate $D$ in Trefethen pg. 54. 
\begin{enumerate}
\item Since $x_i = - x_{n-i}$,
\begin{align}
\label{eqn:firstRelation}
\prod_{k=0,k\neq j}^n (x_j - x_k) = 
(-1)^n\prod_{k=0,k\neq j}^n (x_{n-j} - x_{n-k}) 
= (-1)^n \prod_{k=0,k\neq n-j}^n (x_{n-j} - x_k).
\end{align}
Similarly,
\begin{align}
\label{eqn:secondRelation}
\prod_{k=0,k\neq j,i}^n (x_j - x_k) = 
(-1)^{n-1}\prod_{k=0,k\neq n-j,n-i}^n (x_{n-j} - x_k).
\end{align}
Using relationships \ref{eqn:firstRelation} and 
\ref{eqn:secondRelation},
\begin{align}
\label{eqn:combinedRelation}
D[i,j] = \dfrac{\prod_{k=0,k\neq j,i}^n (x_i - x_k)}
{\prod_{k=0,k\neq j}^n (x_j - x_k)}
= -\dfrac{\prod_{k=0,k\neq n-j,n-i}^n (x_{n-i} - x_k)}
{\prod_{k=0,k\neq n-j}^n (x_{n-j} - x_k)} = - D[n-i,n-j].
\end{align}
\item Sums: $\sum_{k=0}^n x_k = 0$. So, $\sum_{k=0}^n (x_k - x_i) 
= n x_i$. 

\item Products: $a_j := 
	\prod_{k=0,k\neq j}^n (x_j - x_k)$. A result, which 
		we numerically verified, is that $a_j = (-1)^j c$, for some 
		constant $c > 0$ when $j \neq 0,n$. When 
		$j = 0$ or $j = n$, $a_j = (-1)^j 2c$. We use this to simplify Eq. \ref{eqn:DijRaw}:
\begin{align}
	D[i,j] &= 
	\dfrac{\prod_{k=0,k\neq j,i}^n (x_i - x_k)}{\prod_{k=0,k\neq j}^n (x_j - x_k)}\; \quad 
	x = x_i, i \neq j  \\
	&= \dfrac{\prod_{k=0,k\neq j}^n (x_i - x_k)}
	{(x_i - x_j)\prod_{k=0,k\neq j}^n (x_j - x_k)}, 
	\; \; i \neq j\\ 
	&= \begin{cases} 
	\dfrac{2 (-1)^{i+j}}{x_i - x_j}, & j = 0, n \\
		\dfrac{(-1)^{i+j}}{x_i - x_j}, & o.w. 
	\end{cases}
\end{align}
\item Using the previous fact on the products, we 
	can simplify the diagonal elements,
\begin{align}
	D[j,j] &= \sum_{k=0,k\neq j}^n 
	\dfrac{1}{x_j - x_k} =  
	%\sum_{k=0,k\neq j}^n 
	%\dfrac{ }{(-1)^j c}
\end{align}
\end{enumerate}
\end{document}
